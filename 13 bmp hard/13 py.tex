\documentclass{article}
\usepackage[T1,T2A]{fontenc}
\usepackage[utf8]{inputenc}
\usepackage[english,russian]{babel}
\usepackage{amsmath}
\usepackage{amsfonts}
\usepackage{amssymb}
\usepackage{makeidx}
\usepackage{listings}
\usepackage{setspace,amsmath}
\newcommand\tab[1][1cm]{\hspace*{#1}}
\begin{document}

\begin{titlepage}
	\newpage
	
	\begin{center}
		\textbf{Федеральное государственное бюджетное образовательное учреждение высшего образования «Московский государственный университет имени М. В. Ломоносова»}\\
	\end{center}
	
	\vspace{8em}
	
	\begin{center}
		\Large Кафедра вычислительной механики \\ 
	\end{center}
	
	\vspace{2em}
	
	\begin{center}
		\Large \textsc{\textbf{Отчёт по задаче на работу с изображениями по теме:}}
		\\
		\Large \textsc{\textbf{ Изменение яркости BMP-изображения \linebreak}}
	\end{center}
	
	\vspace{15em}
	
	
	
	\begin{flushright}
		\small
		\textbf{Преподаватель: Почеревин Роман Владимирович}\\
		\textbf{Студент 223 группы: Скворцов Андрей Сергеевич}\\
	\end{flushright}
	
	
	\vspace{\fill}
	
	\begin{center}
		Москва \\2024
	\end{center}
	
\end{titlepage}

\begin{center}


{\large\bf Отчёт по работе с BMP-изображениями в Python-3}
\end{center}
\textit{З\,а\,д\,а\,н\,и\,е.} Реализовать алгоритмы фрактального сжатия и восстановления изображения.

 
\textit{Р\,е\,ш\,е\,н\,и\,е.}  Используя библиотеку PIL, можно решить эту задачу эффективнее, чем, например, imageio:

\textsf{from PIL import Image, ImageDraw }

Пусть изображения заданы внутри программы. Выгрузим, посчитаем длину и ширину:
{\usefont{T2A}{cmss}{m}{n}

imin = Image.open("input.bmp")

x1, y1 = imin.size

px1 = imin.load()
}

 Далее созданим выходное изображение и возможность изменять в нем пиксели, а также введем необходимые переменные:
{\usefont{T2A}{cmss}{m}{n}

imout = Image.new("RGB", (x1, y1), (0, 0, 0))
draw = ImageDraw.Draw(imout)

\begin{equation}
size_square = 15
fractal = 0
\end{equation}


 Теперь - логика программы. Найдем среднюю яркость квадратного блока со стороной $size_square$ и запишем ее в переменную $factor $. Затем найдем среднюю яркость 3 соседних блоков того же размера и сохраним их значения в список $factora $.
 
{\usefont{T2A}{cmss}{m}{n}

}

Теперь посмотрим какой из этих 3 блок больше похож на первый сравнивая их по средней яркости и заменяя все 4 блока на самый похожий:

{\usefont{T2A}{cmss}{m}{n}

}
\vspace{1em}

Теперь нарисуем оставшиеся блоки также как в исходном изображени:

{\usefont{T2A}{cmss}{m}{n}

}
\vspace{1em}


Сохраним изображение и очистим память от элемента $draw$:
\vspace{1em}
	
{\usefont{T2A}{cmss}{m}{n}

imout.save("out.bmp", "BMP")

del draw
}
  
\end{document}