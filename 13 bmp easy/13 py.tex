\documentclass{article}
\usepackage[T1,T2A]{fontenc}
\usepackage[utf8]{inputenc}
\usepackage[english,russian]{babel}
\usepackage{amsmath}
\usepackage{amsfonts}
\usepackage{amssymb}
\usepackage{makeidx}
\usepackage{setspace,amsmath}
\newcommand\tab[1][1cm]{\hspace*{#1}}
\begin{document}

\begin{titlepage}
\newpage

\begin{center}
МГУ им. М.Ю.Ломоносова \\
\end{center}

\vspace{8em}

\begin{center}
\Large Кафедра гигачадов \\ 
\end{center}

\vspace{2em}

\begin{center}
\textsc{\textbf{Отчёт по работе с BMP-изображениями в Python-3 \linebreak}}
\end{center}

\vspace{15em}



\newbox{\lbox}
\savebox{\lbox}{\hbox{Почеревин Роман Владимирович}}
\newlength{\maxl}
\setlength{\maxl}{\wd\lbox}
\hfill\parbox{12cm}{
\hspace*{8cm}\hspace*{-5cm}Студент:\hfill\hbox to\maxl{Скворцов Андрей Сергеевич\hfill}\\
\hspace*{8cm}\hspace*{-5cm}Преподаватель:\hfill\hbox to\maxl{Почеревин Роман Владимирович}\\
\\
\hspace*{8cm}\hspace*{-5cm}Группа:\hfill\hbox to\maxl{223}\\
}


\vspace{\fill}

\begin{center}
Москва \\2024
\end{center}

\end{titlepage}

\begin{center}


{\large\bf Отчёт по работе с BMP-изображениями в Python-3}
\end{center}
\textit{З\,а\,д\,а\,н\,и\,е.} Программа должна загрузить изображения из графических файлов
InputFile1 и InputFile2 (размеры входных файлов должны совпадать),
изменить яркость каждого пиксела первого изображения так, чтобы она
равнялась бы яркости соответствующего пиксела второго изображения
и вывести получившееся изображение в графический файл OutputFile.
Под яркостью пиксела подразумевается сумма компонент всех трех компонент пиксела.

 
\textit{Р\,е\,ш\,е\,н\,и\,е.}  Используя библиотеку PIL, можно решить эту задачу эффективнее, чем, например, imageio:

\textsf{from PIL import Image, ImageDraw }

Пусть изображения заданы внутри программы (на примере - "ironman.bmp", "osuzdayu.bmp"). Выгрузим их и посчитаем длину и ширину. На всякий случай, в выходном изображении сделаем минимальную из двух длин и ширин:

{\usefont{T2A}{cmss}{m}{n}

image = Image.open("ironman.bmp")
 
image2 =  Image.open("osuzdayu.bmp") 

draw = ImageDraw.Draw(image) 

width = image.size[0]  

height = image.size[1]  

width2 = image2.size[0]  

height2 = image2.size[1]  

w=min(width, width2)

h=min(height, height2)	

pix = image.load() 

pix2 = image2.load() }


 Теперь - логика программы. Найдем яркость пикселя второго изображения и присвоим ей значение $factor $. Затем найдем яркость пикселя первого изображения и присвоим ей значение $factor1 $.
 
{\usefont{T2A}{cmss}{m}{n}

for i in range(w):

\tab[1cm] for j in range(h):

\tab[2cm]		a = pix2[i, j][0] 

\tab[2cm]		b = pix2[i, j][1] 

\tab[2cm]		c = pix2[i, j][2]

\tab[2cm]		factor = (a+b+c)//3


\tab[2cm]		a1 = pix[i, j][0]  

\tab[2cm]		b1 = pix[i, j][1] 

\tab[2cm]		c1 = pix[i, j][2] 

\tab[2cm]		factor1 = (a1+b1+c1)//3
}

\newpage
Теперь перестроим первое изображение так, чтобы яркость его пикселов равнялась яркости яркости соответствующих пикселов второго. 

{\usefont{T2A}{cmss}{m}{n}

a1 = round(pix[i, j][0] * (factor / factor1))

b1 = round(pix[i, j][1] * (factor / factor1))

c1 = round(pix[i, j][2] * (factor / factor1))

if (a1 < 0):

\tab[1cm]		a1 = 0

if (b1 < 0):

\tab[1cm]		b1 = 0

if (c1 < 0):

\tab[1cm]		c1 = 0

if (a1 > 255):

\tab[1cm]		a1 = 255

if (b1 > 255):

\tab[1cm]		b1 = 255

if (c1 > 255):

\tab[1cm]		c1 = 255

draw.point((i, j), (a1, b1, c1))}
\vspace{1em}


Сохраним изображение и очистим память от элемента $draw$:
\vspace{1em}
	
{\usefont{T2A}{cmss}{m}{n}

image.save("result.bmp", "BMP")

del draw
 }
  
\end{document}